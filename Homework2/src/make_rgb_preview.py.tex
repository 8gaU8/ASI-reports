\begin{lstlisting}[language=python, caption=Make RGB preview from given specral image, label={code:rgb-preview}]
import numpy as np

def search_closest_index(wavelengths: list[float], target_wavelength: float) -> int:
    """Finds the index of the closest wavelength to the target wavelength from value of envi header."""
    closest_index = int(np.argmin(np.abs(np.array(wavelengths) - target_wavelength)))
    return closest_index

def reconstruct_rgb_envi(
    spectral_image: np.ndarray,
    envi_header: dict[str, str],
    rgb_wavelengths: tuple[float, float, float] = (632.15, 528.03, 443.56),
):
    # """Reconstructs RGB image from spectral image with given three wavelengths."""
    # # Get wavelengths from header
    wavelengths = [float(data) for data in envi_header["wavelength"].split(",")]
    rgb_indeces = [search_closest_index(wavelengths, wl) for wl in rgb_wavelengths]
    lines, samples, _bands = spectral_image.shape
    rgb_view = np.empty((lines, samples, 3))
    # Reconstruct RGB image
    for idx, ch in enumerate(rgb_indeces):
        rgb_view[:, :, idx] = spectral_image[:, :, ch] / np.amax(
            spectral_image[:, :, ch]
        )
    return rgb_view

\end{lstlisting}
