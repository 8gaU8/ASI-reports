\section{Figures}
Single figure:
\begin{figure}[H] %
  \centering
  \includegraphics[width=0.5\textwidth]{images/giraffe.jpg}
  \caption[]{Caption}
  \label{fig:label}
\end{figure}

Subfigures:
\begin{figure}[H] %
  \centering
  \subfloat[SubcaptionA]{\includegraphics[width=0.45\textwidth]{images/giraffe.jpg}}
  \hspace{0.1cm}
  \subfloat[SubcaptioB]{\includegraphics[width=0.45\textwidth]{images/giraffe.jpg}}
  \caption[]{Caption}
  \label{fig:label}
\end{figure}

\section{Tables}
You can use this link for tables: \url{https://www.tablesgenerator.com/}

Table:
\begin{table}[H]
  \centering
  %\resizebox{\textwidth}{!} %%%% Activate if it's a long table
  {
    \begin{tabular}{|l|c|r|}
      \hline
      \textbf{Left} & \textbf{Center} & \textbf{Right} \\ \hline \hline
      1 & 2 & 3 \\ \hline
      4 & 5 & 6 \\ \hline
      7 & 8 & 9 \\ \hline

  \end{tabular}}
  \caption{Caption}
  \label{tab:label}
\end{table}

Long table:
\begin{table}[H]
  \centering
  \resizebox{\textwidth}{!}{
    \begin{tabular}{|r|r|r|r|r|r|}
      \hline
      \textbf{TitleVeryVeryLong1} & \textbf{Title2} &
      \textbf{TitleVeryVeryLong3} & \textbf{Title4} &
      \textbf{TitleVeryVeryLong5} & \textbf{Title6} \\ \hline \hline
      1 &2& 3 & 4 & 5  & 6 \\ \hline
      1 &2& 3 & 4 & 5  & 6 \\ \hline

  \end{tabular}}
  \caption{Caption}
  \label{tab:label}
\end{table}

\clearpage
\section{Pseudocode}

Example of simple algorithm:
\begin{algorithm}[H]
  \caption{AlgorithmDescription}
  \label{alg:label}
  \textbf{Input:} Define the input\\
  \textbf{Output:} Define the output

  \begin{algorithmic}[1]
    \STATE Do an operation
    \STATE Do another operation
    \STATE Wait some miliseconds
    \IF{Condition}
    \STATE Do corresponding action
    \ENDIF
    \STATE Do something else
    \STATE \textbf{return} $-1$
  \end{algorithmic}
\end{algorithm}

Example of algorithm with functions predefined:

\begin{algorithm}[H]
  \caption{AlgorithmDescription}
  \label{alg:label}
  \textbf{Input:} Graf $G = (V, E)$ no dirigit i connex, Vector
  d'enters \emph{NND}\\
  \textbf{Output:} \emph{-1} si NND és un conjunt dominador, $i \in
  [0, N)$ altrament
  \begin{algorithmic}[1]
    %For
    \FOR{each $i \in [0, N)$}
    \STATE Do something
    \ENDFOR

    \STATE Explain somehting like "Generakte k+1 neighbors"

    \IF{$NND[i] < \left\lceil
    \displaystyle\frac{G.get\_nb\_veins(i)}{2} \right\rceil$}
    \STATE \textbf{return} $i$
    \ENDIF

    \WHILE{D is not a PIDS}
    \STATE $v* \leftarrow arg max_{\forall v \not\in S } \{funcioGreedy(v)\}$
    \STATE $D \leftarrow  D  \cup  \{v*\}$
    \STATE $updateData(v)$
    \ENDWHILE
    \STATE \textbf{return} $-1$
  \end{algorithmic}
\end{algorithm}

\section{Code fragments}
\begin{lstlisting}[language=Python, caption={CaptionPythonCode}, label={code:label}]
## Insert here the code as it
## Change the language from the headder to adapt the colors

def __main__():
    print("Hello World!")
\end{lstlisting}
